\documentclass[11pt]{article}
\usepackage[utf8]{inputenc} %codificación 
\usepackage[spanish]{babel} %idioma
\usepackage{float}% para usar H
\usepackage{amsmath} %matemáticas
\usepackage{amsthm}
\usepackage{nccmath}
\usepackage{mathrsfs} %% curly letters 
\usepackage{mathtools} %matemáticas
\usepackage{physics} %diferenciales
\usepackage{blindtext}%texto de relleno
\usepackage{amsfonts} %matemáticas
\usepackage{amssymb} %matemáticas
\usepackage{makeidx} 
\usepackage{graphicx} %imágenes
\usepackage[a4paper, margin=2cm, scale=0.8]{geometry}
\usepackage{wrapfig} %figuras alrededor de texto
\usepackage{caption} %subtítulos de figuras
\usepackage{setspace}%interlineado 
\usepackage{array}% entorno array
\usepackage{ragged2e} %alineación 
\usepackage{tabularx} %entorno tabularx (tablas con ancho fijo)
\usepackage{fancyhdr} %cabeceros y pies de página
\usepackage{multirow} %combinar columnas y filas en una tabla
\usepackage{longtable} %mediante este paquete podemos separar una tabla larga en varias que ocupen las páginas necesarias.
\usepackage[table]{xcolor} %con este paquete cambiamos el color de los objetos y concretamente de la opción de las tablas.
\usepackage{enumitem}%enumeraciones personalizadas
\usepackage{subcaption}%subtítulos de figuras y subfiguras
\usepackage{hyperref} % este paquete siempre debemos colocarlo al final 
\usepackage{multicol} %columnas
\usepackage{parskip} %espaciado entre párrafos
\usepackage[rightcaption]{sidecap} %subtítulos alrededor de figuras

% FORMATO PARA FIGURAS
\DeclareCaptionFormat{format1}{
%#1=label, #2 = separator, #3 = text
	\textbf{#1#2}#3
}

\DeclareCaptionFormat{format2}{
%#1=label, #2 = separator, #3 = text
\textsc{#1#2}#3}  

\DeclareCaptionStyle{figura_claro}{
format = format1, 
justification = centering,
font = {color = black,small},
name = Fig.,
width = 0.8\textwidth
}
\DeclareCaptionStyle{figura_claro2}{
	format = format1, 
	justification = centering,
	font = {color = black,small},
	name = Fig.,
	width = 0.5\textwidth
}
\DeclareCaptionStyle{figura_oscuro}{
format = format1, 
justification = centering,
font = {color = white,small},
name = Fig.,
width = 0.8\textwidth
}

\DeclareCaptionStyle{tabla_oscuro}{
format = format1, 
justification = centering,
font = {color = white,small},
name = Tabla,
width = 0.8\textwidth
}

\DeclareCaptionStyle{tabla_claro}{
format = format1, 
justification = centering,
font = {color = black,small},
name = Tabla,
width = 0.8\textwidth
}
%ELECCIÓN DE FORMATO DE FIGURAS
\captionsetup[figure]{style = figura_claro}
\captionsetup[table]{style = tabla_claro}

%FORMATO PARA PÁGINAS OSCURAS Y CLARAS
\fancypagestyle{claroi}{
\fancyhf[]{}
\fancyhead[L]{\color{black}\textsc{UAM}}
\fancyhead[R] {\color {black}\textsc{Electrodinámica Clásica - Unidad 2}}
\fancyfoot[C]{\color{black}\thepage}
\setlength{\headheight}{18pt} 
\renewcommand{\headrulewidth}{0.75pt}
\renewcommand{\headruleskip}{-0.5em}
\renewcommand{\headrule}{\hbox to\headwidth{%
    \color{black}\leaders\hrule height \headrulewidth\hfill}}
  \renewcommand{\footrulewidth}{0pt}
}


\fancypagestyle{oscuroi}{
\fancyhf[]{}
\fancyhead[L]{\color{white}\textsc{UAM}}
\fancyhead[R] {\color {white}\textsc{Joan Andrés Mercado Tandazo}}
\fancyfoot[C]{\color{white}\thepage}
\renewcommand{\headrulewidth}{0.75pt}
\renewcommand{\headruleskip}{-0.5em}
\renewcommand{\headrule}{\hbox to\headwidth{%
    \color{white}\leaders\hrule height \headrulewidth\hfill}}
  \renewcommand{\footrulewidth}{0pt}
}
\def\ftoscuro{
\pagestyle{fancy}
\fancyfoot[]{}
\fancyfoot[C]{\color{white}\thepage}
}
\def\ftclaro{
\pagestyle{fancy}
\fancyfoot[]{}
\fancyfoot[C]{\color{black}\thepage}
}
% NO SE SUELE USAR ESTO
\def\hdoscuro{
\pagestyle{fancy}
\fancyhead[]{}
\fancyhead[L]{\color{white}\textsc{UAM}}
\fancyhead[R] {\color {white}\textsc{Joan A. Mercado}}
\renewcommand{\headrulewidth}{0.75pt}
}
\def\hdclaro{
\pagestyle{fancy}
\fancyhead[]{}
\fancyhead[L]{\color{black}\textsc{UAM}}
\fancyhead[R] {\color {black}\textsc{Electrodinámica Clásica - Unidad 1}}
}

% ELECCIÓN MODO OSCURO O CLARO
\def\darkmode{
\pagecolor{black}
\color{white}
\pagestyle{oscuroi}
\captionsetup[figure]{style=figura_oscuro}
\captionsetup[table]{style = tabla_oscuro}
}
\def\brightmode{
\pagecolor{white}
\color{black}
\pagestyle{claroi}
\captionsetup[figure]{style=figura_claro}
\captionsetup[table]{style = tabla_claro}
}
%QUITAR NÚMERO EN UNA PÁGINA
\def\resetnumpagetitle{
\thispagestyle{empty}
\newpage
\setcounter{page}{1}
}
\DeclareMathOperator{\rot}{\textrm{\textbf{rot}}}
\DeclareMathOperator{\diver}{\textrm{\textbf{div}}}
\newcommand{\FT}[1]{\mathcal{F}\{#1\}}
\newcommand{\LT}[1]{\mathcal{L}\left\{ #1 \right\}}
\newcommand{\BLT}[1]{\mathcal{B}\left\{#1\right\}}
\renewcommand{\grad}{\textrm{\textbf{grad}}}


\definecolor{coolgreen}{RGB}{15, 219, 97}


\hypersetup{
colorlinks = true,%% atento a la separación con comas
%% si colocas colorlinks = false aparecen cajas alrededor de los links pero no se ven ni con texmaker ni los navegadores convencionales,en cambio sí que se ve con adobe acrobat reader. 
linkcolor = blue, 
citecolor= blue,
filecolor = magenta, 
urlcolor = blue,
pdfpagemode = FullScreen,
urlbordercolor = {1 0 0},
linktocpage = false, %% si es verdadero son las páginas del índice las que quedan referenciadas
}
\urlstyle{same}

\author{\Large \Large Subgrupo 4:\\ 
\Large Juan Manual Sánchez Arrua\\
 \Large Jaime Sánchez-Carralero Morato\\
 \Large Óscar Marzal Bardón\\
 \Large Joan Andrés Mercado Tandazo}
\date{\Large UAM-ELECTRODINÁMICA CLÁSICA}
\title{\huge \textbf{Cuarta Entrega}}
\begin{document}
\maketitle
\brightmode
\section*{Problema 7}
Trabajando en forma tridimensional y partiendo de la transformación
de Lorentz del espacio-tiempo entre sistemas inerciales K y K' que
se desplazan según el eje X, deducir de la covariancia Lorentz de las
ecuaciones homogéneas de Maxwell la transformación del campo electromagnético.
\vspace*{1em}
\hrule
\vspace*{1em}
{\color{blue} Resolución:}\\
Las ecuaciones homogéneas de Mawxell son:
\begin{subequations}
	\begin{empheq}[left=\empheqlbrace]{align}
		\nabla \cdot \vec{B} &= 0\\
		\nabla \times \vec{E} &= -\dfrac{1}{c}\pdv{\vec{B}}{t}
	\end{empheq}
\end{subequations}

En componentes se tiene:
\begin{subequations}\label{eq:Maxwell K}
	\begin{empheq}[left=\eqref{eq:Maxwell K}\;\empheqlbrace]{align}
		\pdv{B_x}{x} + \pdv{B_y}{y} + \pdv{B_z}{z} = 0\label{Gauss-K}\\
		\pdv{E_z}{y} - \pdv{E_y}{z} = -\dfrac{1}{c}\pdv{B_x}{t}\label{Faraday-1-K}\\
		\pdv{E_x}{z} - \pdv{E_z}{x} = -\dfrac{1}{c}\pdv{B_y}{t}\label{Faraday-2-K}\\
		\pdv{E_z}{y} - \pdv{E_y}{z} = -\dfrac{1}{c}\pdv{B_z}{t}\label{Faraday-3-K}
	\end{empheq}
\end{subequations}
Por otro lado, las transformaciones de Lorentz entre los sistemas $K$ y $K'$:
\begin{subequations}
	\begin{empheq}[left=\empheqlbrace]{align}
		x' &= \gamma \qty(x + \beta ct)\\
		ct'&= \gamma \qty (ct + \beta x)\\
		y' &= y\\
		z' &= z
	\end{empheq}
\end{subequations}
Esto nos lleva a la siguiente relación entre operadores diferenciales:

\begin{subequations}\label{eq:partials}
	\begin{empheq}[left=\eqref{eq:partials}\;\empheqlbrace]{align}
		\pdv{x'} &= \gamma \qty(\pdv{x} + \dfrac{\beta}{c}\pdv{t})\label{eq:partial x-x'}\\
		\pdv{t'} &= \gamma \qty(\pdv{t} + \beta c \pdv{x}) \label{eq:partial t-t'}\\
		\pdv{y'} &= \pdv{y} \label{eq: partial y-y'}\\
		\pdv{z'} &= \pdv{z} \label{eq: partial z-z'}
	\end{empheq}
\end{subequations}
Imponiendo covarianza con las transformaciones de Lorentz se debe tener que en el sistema $K'$ se debe tener la siguiente forma de las ecuaciones de Maxwell:
\begin{subequations}\label{eq:Maxwell K'}
	\begin{empheq}[left=\eqref{eq:Maxwell K'}\;\empheqlbrace]{align}
		\pdv{B'_x}{x'} + \pdv{B'_y}{y'} + \pdv{B'_z}{z'} = 0 \label{eq:Gauss}\\
		\pdv{E'_z}{y'} - \pdv{E'_y}{z'} = -\dfrac{1}{c}\pdv{B'_x}{t'}\label{eq:Faraday-1}\\
		\pdv{E'_x}{z'} - \pdv{E'_z}{x'} = -\dfrac{1}{c}\pdv{B'_y}{t'}\label{eq:Faraday-2}\\
		\pdv{E'_y}{x'} - \pdv{E'_x}{y'} = -\dfrac{1}{c}\pdv{B'_z}{t'}\label{eq:Faraday-3}
	\end{empheq}
\end{subequations}
Aplicando las ec. \eqref{eq:partials} sobre la ec.\eqref{eq:Gauss}:
\begin{equation}
	\gamma\pdv{B'_x}{x} + \gamma \dfrac{\beta}{c}\pdv{B'_x}{t} + \pdv{B'_y}{y} + \pdv{B'_z}{z} = 0 \label{eq:1}
\end{equation}
Podemos aplicar asimismo las ec. \eqref{eq:partials} sobre la componente $x$ de la ley de Faraday \eqref{eq:Faraday-1}:
\begin{gather}
	\partial_y E_z - \partial_z E_y = -\dfrac{\gamma}{c}\qty(\partial_t B'_x + \beta c \partial_x B'_x)\nonumber\\
	\partial_y E_z - \partial_z E_y  + \gamma\beta \partial_x B'_x = -\dfrac{\gamma}{c} \qty(\partial_t B'_x)  \overset{\eqref{eq:1}}{\;\Rightarrow\;}\nonumber\\
	\;\Rightarrow\; \partial_y\qty(E'_z -\beta B'_y) - \partial_z\qty(E'_y + \beta B'_z) = -\dfrac{\gamma}{c}\partial_t B'_x (1-\beta^2 )\nonumber\\
	\partial_y\qty(\gamma\qty(E'_z - \beta B'_y)) - \partial_z\qty(\gamma \qty(E'_y + \beta B'_z))= -\dfrac{1}{c} \partial_t B'_x \label{eq:Faraday-1-fin}
\end{gather}
De la misma forma podemos operar sobre la componente y de la ley de Faraday \eqref{eq:Faraday-2}:
\begin{gather}
	\partial_zE'_x - \gamma\qty(\partial_x + \dfrac{\beta}{c}\partial_t)E'_z  = -\dfrac{\gamma}{c}\qty(\partial_t  + \beta c \partial_x)B'_y\nonumber\\
	\partial_z E'_x - \partial_x\qty(\gamma\qty(E'_z - \beta B'_y))=\dfrac{1}{c}\partial_t(\gamma\qty(B'_y - \beta E'_z))\label{eq:Faraday-2-fin}
\end{gather}
Y finalmente sobre la componente z de la ley de Faraday \eqref{eq:Faraday-3}:
\begin{gather}
	\gamma\qty(\partial_x +\dfrac{\beta}{c}\partial_t)E'_y - \partial_y E'_x = -\dfrac{\gamma}{c} \qty (\partial_t + \beta c\partial_x)B'_z\nonumber\\
	\partial_x\qty(\gamma\qty(E'_y + \beta B'_z))-\partial_y E'_x = -\dfrac{1}{c}\partial_t\qty(\gamma\qty(B'_z + \beta E'_y))\label{eq:Faraday-3-fin}
\end{gather}
De esta forma, comparando las expresiones \eqref{eq:Faraday-1-fin}-\eqref{eq:Faraday-3-fin} con las ecuaciones de Maxwell en el sistema K' \eqref{eq:Maxwell K'} se tiene que la transformación del campo electromagnético entre los sistemas es de la siguiente forma:
\begin{subequations}
    \begin{empheq}[left=\empheqlbrace]{align}
        	E_x &= E'_x\\
        	B_x &= B'_x\\
        	E_y &= E'_y +\beta B'_z\\
        	B_y &= B'_y - \beta E'_z\\
        	E_z &= E'_z -\beta B'_y\\
        	B_z &= B'_z + \beta E'_y\quad\qed
    \end{empheq}
\end{subequations}

\end{document}